\documentclass[a4paper]{article}
\usepackage[T1]{fontenc}
\usepackage[utf8]{inputenc}
\usepackage[italian]{babel}
\usepackage{booktabs}
\usepackage{caption}
\usepackage[table,xcdraw]{xcolor}
\usepackage{wrapfig}
\usepackage{graphicx}
\usepackage[margin=1in]{geometry} 
\usepackage{fancyvrb}

\title{%
  Sistemi Operativi \\
  \large Riassunto delle lezioni di Laboratorio A.A. 2018/2019}

\author{Matteo Franzil}

\begin{document}

\maketitle

\tableofcontents

\section{Comandi Bash}

La seguente è una lista, non esaustiva, dei comandi visti a lezione per l'uso della shell Bash.

\subsection{Comandi}
\begin{itemize}
\item\verb|free| $\rightarrow$ memoria libera
\item\verb|pwd| $\rightarrow$ cartella corrente
\item\verb|df| $\rightarrow$ partizioni
\item\verb|dirname| $\rightarrow$ ottiene il nome della cartella di un dato file 
\end{itemize}

\subsection{Piping}
\begin{itemize}
\item\verb|;| $\rightarrow$ esecuzione in sequenza
\item\verb=&& ||= $\rightarrow$ esecuzione in sequenza con corto circuito 
\item\verb=|= $\rightarrow$ piping classico
\item\verb|>| $\rightarrow$ redirect su file classico
\item\verb|1>| $\rightarrow$ redirect di \verb|stdout|
\item\verb|2>| $\rightarrow$ redirect di \verb|stderr|
\end{itemize}

\subsection{Intestazioni}
\begin{itemize}
\item\verb|#! /bin/bash| $\rightarrow$ intestazione
\item\verb|#Script "Hello World"| $\rightarrow$ intestazione nominativa
\end{itemize}

\subsection{Scripting}
\begin{itemize}
\item\verb|var=VALORE| $\rightarrow$ assegnazione (sono sempre stringhe)
\item\verb|${var}| $\rightarrow$ stampa (con eventuale esecuzione)
\item\verb|$@| $\rightarrow$ equivalente ad \verb|argv|
\item\verb|$#| $\rightarrow$ equivalente ad \verb|argc|
\item\verb|$1|, \verb|$2|, \verb|$3|, \verb|$n| $\rightarrow$ i primi 9 parametri passati
\item\verb|shift| $\rightarrow$ cestina il primo argomento nella lista
\item\verb|" ... "| $\rightarrow$ crea un comando / stringa interpolando variabili
\item\verb|$(( ... ))| $\rightarrow$ contengono espressioni aritmetiche: se all'interno uso una \verb|$var|, viene sostituita come fosse una macro, se uso il singolo contenuto di \verb|var| allora viene inserito il valore come avesse le parentesi.
\item\verb|bc| $\rightarrow$ comando che supporta il piping in entrata, per eseguire operazioni in float
\item\verb|# Commenti| $\rightarrow$ commento classico di singola riga
\item\verb|$?| $\rightarrow$ valore di ritorno globale, usato dagli script (tipo return 0 in C): ha significato booleano (\textbf{0} niente errori, arriva fino a \textbf{256})
\end{itemize}

\subsection{Booleani}
\begin{itemize}
\item\verb|test ...| $\rightarrow$ si aspetta un espressione booleana, e internamente modifica il registro booleano visto prima:
\item\verb|-eq, -ne, -lt, -gt| $\rightarrow$ operandi booleani utilizzati
\item\verb|[ ... ]| $\rightarrow$ sintassi di testing alternativa (gli spazi sono importanti!); attenzione che le parentesi quadrate sono considerate come ultimo comando eseguito
\item\verb|[[ ... ]]| $\rightarrow$ raggruppamento di espressioni booleani per utilizzare operatori comuni (>, < ...)- 
\item\verb| -f (file) -d (directory)| $\rightarrow$ verificano l'esistenza di un dato file/cartella. 
\end{itemize}

\subsection{Cicli}
\begin{itemize}
\item\verb|if [ ... ]; then ... ; else ...; fi| $\rightarrow$ costrutto if standard
% Utilizzo = al posto di un pipe per evitare conflitti inutili
\item\verb=case $var in; a|b) ... ;; c) ... ;; esac= $\rightarrow$ costrutto switch standard
\item\verb|for (( init ; case; step )); do; ...; done| $\rightarrow$ costrutto for standard
\item\verb|until [[ ... ]] ; ...; done| $\rightarrow$ while negato
\item\verb|while [[ ... ]] ; ...; done| $\rightarrow$ while standard
\end{itemize}
 
\subsection{Funzioni}
\begin{itemize}
\item\verb|func() { ... }| $\rightarrow$ accedibile come fossero degli script (\verb|func arg1 ... argn|)
\end{itemize}

\subsection{Varie}
\begin{itemize}
\item\verb|$( ... )| $\rightarrow$ sottoshell che esegue comandi in un processo separato
\item\verb|BASH_SOURCE[0]| $\rightarrow$ contiene il nome dello script in esecuzione
\item\verb|exit n| $\rightarrow$ uscita con codice d'errore
\item\verb|1> 2> ... n>| $\rightarrow$ redirezionamento dei diversi canali sui file, come visto prima (1 = stdout, 2 = stderr); è possibile redirezionare stderr su stdout e viceversa tramite il comando \verb|2>&1|, oppure mettendo tutto su file: \verb|1>output 2>&1|. L'ordine in cui vengono interpretati i redirect sono da destra a sinistra.
\end{itemize}

\section{Make}

\subsection{Introduzione}

Make è un tool utilizzato per automatizzare processi all'interno di sistemi Unix. Viene principalmente usato per automatizzare la compilazione dei file.

I Makefile sono composti da regole, composte da un \textbf{identificativo} (o nome) e da una \textbf{ricetta} (una serie di comandi indentati con una tabulazione \verb|\t| rispetto al nome della ricetta):

\begin{verbatim}
regola:
    echo "Ciao"
\end{verbatim}

I file vengono poi eseguiti tramite il comando \verb|make -f nome_file.makefile|. Notare come i comandi vengono anche stampati su \verb|stdout| oltre a essere eseguiti. Questa funzionalità può essere sfruttata per stampare a video anche i procedimenti che vengono eseguiti dal file, ma può risultare fastidioso: si può quindi inserire una chiocciola \verb|@| che impedisce la stampa del comando.

Il nome del file è opzionale e se non presente viene cercato all'interna della cartella corrente un nome corrispondente a \verb|Makefile|.

In ogni ricetta, ogni riga è trattata singolarmente come un singolo processo, caricando la giusta cartella di lavoro di volta in volta. Bisogna quindi fare attenzione all'uso di \verb|cd|.

All'inizio di ogni regola, si può inoltre specificare una o più \textit{dipendenze} che devono essere rispettate prima di avviare la regola corrispondente. A tal scopo si usa generalmente dare il nome del file che verrà generato alla regola stessa, se la regola ne genera qualcuno:

\begin{verbatim}
file.cc: dipendenza1 dipendenza2... 
    comandi per generare file.cc...
\end{verbatim}

In tutti gli altri casi si parla di \textit{pseudo-regole} ed il nome può essere attribuito di fantasia.

\subsection{Definizione di macro}

All'interno dei makefile si possono definire delle macro, che possono essere utilizzate con una sintassi analoga a bash. Ciò risulta problematico in quanto \verb|$| è interpretato prima da Makefile e poi da Bash.

\begin{verbatim}
MACRO1=Pippo
main:
    @echo Hello $(MACRO1)
\end{verbatim}

Le macro possono inoltre essere sovrascritte al momento dell'avvio di \verb|MAKE| con la medesima sintassi (ma è necessario che la macro stessa sia definita all'interno del file). 

Con la sintassi \verb|$(...)| è possibile accedere a una sotto-shell, analoga a quella di Bash, per eseguire comandi all'interno del Makefile stesso. In questo sotto-comando tutti gli \verb|\n| vengono tramutati in spazi.

\section{gcc}

\subsection{Introduzione}

\verb|gcc| è un compilatore multi source/target per compilare i file sorgenti C. Viene utilizzato in combinazione con Make per compilare velocemente più sorgenti.

\subsection{Flag}
\begin{itemize}
\item\verb|gcc main.c -S| $\rightarrow$ compila in codice Assembly
\item\verb|gcc main.c -E| $\rightarrow$ esegue solo il preprocessore
\item\verb|gcc main.c -c| $\rightarrow$ compila senza linkare
\item\verb|gcc main.c -o| $\rightarrow$ genera i file oggetto binari
\end{itemize}

\end{document}
